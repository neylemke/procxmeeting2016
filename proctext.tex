
\chapter{Organizing Committee}

\begin{description}

\item[AB3C President:] Glória R Franco (UFMG)

\item[AB3C Vice President:] Alan M Durham (USP)

\item[BSB Chair:] Sérgio Campos (UFMG)

\item[AB3C Secretaries]:

\begin{itemize}
 \item Marcelo Brandão (Unicamp) 
\item  Ney Lemke (Unesp)
\end{itemize}

\item[AB3C Financial Department]:

\begin{itemize}
\item Priscila Grynberg (Embrapa)
\item Fábio Passetti (Fiocruz)
\end{itemize}

\item[Poster Session Organizers]:

\begin{itemize}
\item Mainá Bitar (UFMG)
\item Nicole Scherer (INCA)
\end{itemize}

\item[Paper Submission Organizers]:  

\begin{itemize}
\item Sérgio Campos (UFMG)
\item Marcelo Brandão (Unicamp)
\item Ney Lemke (Unesp)
\item André Fujita (USP)
\item Ronnie Alves (Université Montpellier, França)
\end{itemize}

\item[Local Committee]:

\begin{itemize}
\item Alan Durham (USP)
\item André Fujita (USP)
\item Arthur Gruber (USP)
\item  Ronaldo Fumio Hashimoto (USP)
\end{itemize}
\end{description}
\newpage
\chapter{Introduction}
The Brazilian Association of Bioinformatics and Computational Biology (AB3C) is a scientific society funded in July 12th 2004. Since its creation, AB3C has been responsible for the annual conference entitled “X-Meeting” which is the main Bioinformatics and Computation Biology event in Brazil. This year its 11th edition will be held in São Paulo, the biggest city in South America.

	
Bioinformatics is now a strategic area for Brazil and all Latin America and, therefore, it is also strategic to the development of Science, Technology and Economy. The X-Meeting is a Brazilian event with international reach which has an average of 400 participants. The Conference is an opportunity for students, researchers and companies to interact and difuse knowledge. The AB3C has been a pioneer society in the field of Bioinformatics in Brazil and we have a history of ten past very productive meetings.